%%%%%%%%%%%%%%%%%%%%%%%%%%%%%%%%%%%%%%%%%%%%%%%%%%%%%%%%%%%%%%%%%%%%
%% I, the copyright holder of this work, release this work into the
%% public domain. This applies worldwide. In some countries this may
%% not be legally possible; if so: I grant anyone the right to use
%% this work for any purpose, without any conditions, unless such
%% conditions are required by law.
%%%%%%%%%%%%%%%%%%%%%%%%%%%%%%%%%%%%%%%%%%%%%%%%%%%%%%%%%%%%%%%%%%%%

\documentclass[
  digital, %% This option enables the default options for the
           %% digital version of a document. Replace with `printed`
           %% to enable the default options for the printed version
           %% of a document.
  table,   %% Causes the coloring of tables. Replace with `notable`
           %% to restore plain tables.
  nolof,     %% Prints the List of Figures. Replace with `nolof` to
           %% hide the List of Figures.
  nolot,     %% Prints the List of Tables. Replace with `nolot` to
           %% hide the List of Tables.
  twoside,
  nocover,
  monochrome,
  12pt
  %% More options are listed in the user guide at
  %% <http://mirrors.ctan.org/macros/latex/contrib/fithesis/guide/mu/fi.pdf>.
]{fithesis3}
%% The following section sets up the locales used in the thesis.
\usepackage[resetfonts]{cmap} %% We need to load the T2A font encoding
\usepackage[T1,T2A]{fontenc}  %% to use the Cyrillic fonts with Russian texts.
\usepackage[
  main=czech, %% By using `czech` or `slovak` as the main locale
                %% instead of `english`, you can typeset the thesis
                %% in either Czech or Slovak, respectively.
  english, german, russian, slovak %% The additional keys allow
]{babel}        %% foreign texts to be typeset as follows:
%%
%%   \begin{otherlanguage}{german}  ... \end{otherlanguage}
%%   \begin{otherlanguage}{russian} ... \end{otherlanguage}
%%   \begin{otherlanguage}{czech}   ... \end{otherlanguage}
%%   \begin{otherlanguage}{slovak}  ... \end{otherlanguage}
%%
%% For non-Latin scripts, it may be necessary to load additional
%% fonts:
\usepackage{paratype}
\def\textrussian#1{{\usefont{T2A}{PTSerif-TLF}{m}{rm}#1}}
%%
%% The following section sets up the metadata of the thesis.
\thesissetup{
    university    = mu,
    faculty       = fi,
    type          = mgr,
    author        = Bc. Tomáš Skopal,
    gender        = m,
    advisor       = RNDr. Filip Nguyen,
    title         = {Distributed Complex Event Processing},
    TeXtitle      = {Distributed Complex Event Processing},
    keywords      = {keyword1, keyword2, ...},
    TeXkeywords   = {keyword1, keyword2, \ldots},
    date			 =	2016/05/30
}
\thesislong{abstract}{
    Goal of this thesis is to develop a Peer to Peer algorithm for distributed Event Pattern matching.. The application should be able to run any number of processing nodes. For the needs of this thesis, example of 4 nodes will be sufficient.
}

%% The following section sets up the bibliography.
\usepackage{csquotes}
\usepackage[              %% When typesetting the bibliography, the
  backend=biber,          %% `numeric` style will be used for the
  style=numeric,          %% entries and the `numeric-comp` style
  citestyle=numeric-comp, %% for the references to the entries. The
  sorting=none,           %% entries will be sorted in cite order.
  sortlocale=auto         %% For more unformation about the available
]{biblatex}               %% `style`s and `citestyles`, see:
%% <http://mirrors.ctan.org/macros/latex/contrib/biblatex/doc/biblatex.pdf>.
\addbibresource{example.bib} %% The bibliograpic database within
                          %% the file `example.bib` will be used.
\usepackage{makeidx}      %% The `makeidx` package contains
\makeindex                %% helper commands for index typesetting.
%% These additional packages are used within the document:
\usepackage{paralist}
\usepackage{amsmath}
\usepackage{amsthm}
\usepackage{amsfonts}
\usepackage{url}
\usepackage{menukeys}
\begin{document}
\chapter{Úvod}
Cílem této práce je vytvoření  ...

\chapter{Zpracování událostí}
Se zvyšujícím se počtem zařízení, která jsou schopna produkovat data, se zvyšuje potřeba tato data analyzovat. Běžně rozšířeným způsobem je zpracování dat dávkově. Tedy, data se uloží a ve vhodnou dobu, typicky v noci, se analyzují.

Pokud však uvažujeme reálný provoz na síti, který se dnes v centrálních uzlech pohybuje okolo $1 Tb/s$ 
%%[https://is.muni.cz/do/rect/habilitace/1433/44368572/44368651/HP_kor.-verejna.pdf]
, je dávkové zpracování nereálné. Potřebujeme data analyzovat za běhu (angl. real time).

Jednotkou zpracování dat je událost (angl. event). Událost je základním pojmem používaným v oblasti zpracování událostí. Je definována jako objekt, který reprezentuje záznam o aktivitě v daném systému. Událost může mít vlastnosti. Typickým příkladem takové vlastnosti je čas vzniku události, příčina jejího vzniku nebo její typ. [\ref{bib_1}] Jednoduchým příkladem události může být paket. Je to datová schránka, která obsahuje informace, které můžeme analyzovat. Samostatný paket nemá téměř žádnou vypovídající hodnotu, kdežto proud paketů je základem Internetu.

Takový proud událostí skrývá množství dat, která je možné získat až při komplexní analýze, která zohledňuje více událostí v řadě. To nazýváme \textit{komplexní zpracování dat (angl. complex event processing neboli CEP)}


\section{CEP}

Je těžké shrnout celý vědní obor pod jednu všeobjímající definici. David Luckham ve své knize  THE POWER OF EVENTS: AN INTRODUCTION TO COMPLEX EVENT PROCESSING IN DISTRIBUTED ENTERPRISE SYSTEMS [\ref{bib_1}] říká, že CEP je soubor technik a nástrojů, které pomáhají k pochopení a kontrole událostmi řízených systémů.

Jak už bylo řečeno, množství událostí v systémech je enormní. Při jejich zpracování se setkáváme s pojmem \textit{komplexní událost}. Taková událost se může vyskytnout pouze jako reakce na sled jiných, dílčích, událostí. Dílčí události mohou spolu souviset mnoha různými způsoby, nejčastěji je však spojujeme na základě vlastností (čas vzniku, příčina vzniku, typ, atd).

Zde bude popis komplexni udalosti "last touch" z Facebooku

TODO: pouze prelozit: One of the major themes of CEP is that different kinds of events are related. CEP provides techniques for defining and utilizing relationships between events. CEP applies to any type of event that happens in a computer application or a network or an information system. In fact, one of its techniques lets you define your own events as patterns of the events in your computer system. CEP lets you see when your events happen. This is one way to understand what is going on in your system.

That brings us to another point—flexibility. CEP allows users to specify the events that are of interest to them at any moment. Events of interest can be low-level network monitoring alerts or high-level enterprise management intelligence, depending upon the role and viewpoint of individual users. Different kinds of events can be specified and monitored simultaneously. And the specification of the events of interest, how they should be viewed and acted upon, can be changed on the fly, while the system is running.

Musím navázat na následující kapitolu, tím, že v dnešní době už nestačí analyzovat data pouze na jednom PC, ale v clusteru.
\section{Distribuované CEP}
Zde chci popsat co to je distribuované zpracování dat. V další kapitole pak navážu, tím jaké nástroje v současnosti pro zpracování dat máme (Samsa, Storm)
\chapter{Nástroje pro distribuované zpracování událostí}
\section{Apache Samsa}
\section{Apache Storm}
\section{CAVE}
http://dl.acm.org/citation.cfm?doid=2675743.2771834
\chapter{Analýza a návrh}
se lehce dostknu toho co zminenym technologiim chybi a co by navrhovane reseni (ktere budu popisovat dale) umoznilo delat jinak. Bude to priprava ctenare na to co prijde.
 Mely by zde byt take nastineny vstupy a vystupy a obecny navrh reseni. Bez znalosti toho jak to konkretne bude udelane. Defacto interface. Zde bude zakomponovan i prepis filipova zadani
\chapter{Vytvoření clusteru v rámci sítě}
\section{Motivace}
\section{Technologie}
Tato kapitola popisuje jednotlivé technologie, které jsou použity při implementaci algoritmu. Na konci každé subsekce bude popis toho jak konkrétně je technologie použita v mém řešení.
\subsection{Apache Maven}
\subsection{Apache Kafka}
\subsection{Apache ZooKeeper}
Popis technologie podle dokumentace

Konkrétní použití: Jak se sestavuje strom a k čemu slouží (je důležité zmínit, že jde pouze o virtuální stav a je nutné myslet na to, že aplikace běží na daném uzlu pouze jednou). Jaké má zookeeper tree výhody (dá se zjistit jestli a jaké má uzel potomky, dá se kterýkoli uzel modifikovat, což způsobí příslušnou změnu v aplikaci).

CuratorFramework
\subsection{Esper}
\section{Události v systému}
Kapitola popisující hlavní myšlenku povahy dat, která bude algoritmus vyhodnocovat. Také zde budou ukázky používaných dat.
\subsection{Hrubozrnné}
\subsection{Jemnozrnné}
\section{Konfigurace}
\section{Implementace}
Pro implementaci byla zvolena Java (konkrétně ve verzi 1.8), protože všechny použité technologie mají dobrá API pro Javu a většina příkladů je právě v Javě. Dalším důvodem je také to, že Java se dobře hodí pro běh aplikací tohoto druhu, protože má dobrou práci s vlákny. Posledním, méně důležitým, důvodem je popularita Javy a snadná čitelnost kódu.
\subsection{Architektura aplikace}
Popis jednotlivých maven modulů. K čemu který slouží
\subsection{Iniciální spuštění}
\subsection{Přechody mezi stavy}
Jedna z nejdůležitějších kapitol, která ukazuje co způsobí, že aplikace začne vyhodnocovat jemnozrnné události. Počínaje tím, že Esper vyhodnotí proud událostí a emituje ep-událost. Dále přes zpracování ep-události, nastavení příslušných dat jednotlivým zk-uzlům v zk-stromě, či modifikaci zk-stromu. Až po ukončení zpracovávání jemnozrnný událostí a návrat k iniciálnímu stavu.

Bude zde také zmíněno dynamické nasazování nových ep pravidel. To by se dalo shrnout pod nadpis "ovládání clusteru z venčí" - řešeno přes nastavování dat jednotlivým uzlům v zookeeperu.

\subsection{Esper pravidla}
Mini kapitola, kterou bych věnoval použitým esper pravidlům.
\section{Demo}
Nějaké print screeny. Jednoduše ukázka běhu programu.
\section{Známá omezení}
Diskuse nedostatků nebo možných vylepšení výše navrženého řešení. 
\begin{itemize}
	\item Momentální nemožnost spustit více consumerů na jednom stroji.
	\item Velmi náročný monitoring a spouštění jednotlivých uzlů.
	\item Zatím nevím jak je to s dynamickým přidáváním nových uzlů do hierarchie.
	\item Nejsou vůbec otestovány výpadky některých uzlů. Zookeeper to zvládne, kafka také, ale co se stane s virtuálním zk-tree v aplikaci?
\end{itemize}

\chapter{Závěr}
Závěr bude v tomto případě obsahovat obšírnější zhodnocení toho jak se povedlo splnit zadání. Že výsledkem práce je navržené řešení za použití kafky, zk, esperu, Javy.


\begin{thebibliography}{99}
\bibitem
LLUCKHAM, David. \textit{The Power of Events: An Introduction to Complex
Event Processing in Distributed Enterprise Systems.} Pearson
Education, Inc., 2002, ISBN 9780201727890 \label{bib_1}
\end{thebibliography}

\appendix %% Start the appendices.

\end{document}

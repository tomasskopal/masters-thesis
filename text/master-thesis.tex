%%%%%%%%%%%%%%%%%%%%%%%%%%%%%%%%%%%%%%%%%%%%%%%%%%%%%%%%%%%%%%%%%%%%
%% I, the copyright holder of this work, release this work into the
%% public domain. This applies worldwide. In some countries this may
%% not be legally possible; if so: I grant anyone the right to use
%% this work for any purpose, without any conditions, unless such
%% conditions are required by law.
%%%%%%%%%%%%%%%%%%%%%%%%%%%%%%%%%%%%%%%%%%%%%%%%%%%%%%%%%%%%%%%%%%%%

\documentclass[
  digital, %% This option enables the default options for the
           %% digital version of a document. Replace with `printed`
           %% to enable the default options for the printed version
           %% of a document.
  table,   %% Causes the coloring of tables. Replace with `notable`
           %% to restore plain tables.
  nolof,     %% Prints the List of Figures. Replace with `nolof` to
           %% hide the List of Figures.
  nolot,     %% Prints the List of Tables. Replace with `nolot` to
           %% hide the List of Tables.
  twoside,
  nocover,
  monochrome,
  12pt
  %% More options are listed in the user guide at
  %% <http://mirrors.ctan.org/macros/latex/contrib/fithesis/guide/mu/fi.pdf>.
]{fithesis3}
%% The following section sets up the locales used in the thesis.
\usepackage[resetfonts]{cmap} %% We need to load the T2A font encoding
\usepackage[T1,T2A]{fontenc}  %% to use the Cyrillic fonts with Russian texts.
\usepackage[
  main=czech, %% By using `czech` or `slovak` as the main locale
                %% instead of `english`, you can typeset the thesis
                %% in either Czech or Slovak, respectively.
  english, german, russian, slovak %% The additional keys allow
]{babel}        %% foreign texts to be typeset as follows:
%%
%%   \begin{otherlanguage}{german}  ... \end{otherlanguage}
%%   \begin{otherlanguage}{russian} ... \end{otherlanguage}
%%   \begin{otherlanguage}{czech}   ... \end{otherlanguage}
%%   \begin{otherlanguage}{slovak}  ... \end{otherlanguage}
%%
%% For non-Latin scripts, it may be necessary to load additional
%% fonts:
\usepackage{paratype}
\def\textrussian#1{{\usefont{T2A}{PTSerif-TLF}{m}{rm}#1}}
%%
%% The following section sets up the metadata of the thesis.
\thesissetup{
    university    = mu,
    faculty       = fi,
    type          = mgr,
    author        = Bc. Tomáš Skopal,
    gender        = m,
    advisor       = RNDr. Filip Nguyen,
    title         = {Distributed Complex Event Processing},
    TeXtitle      = {Distributed Complex Event Processing},
    keywords      = {keyword1, keyword2, ...},
    TeXkeywords   = {keyword1, keyword2, \ldots},
    date			 =	2016/05/30
}
\thesislong{abstract}{
    Goal of this thesis is to develop a Peer to Peer algorithm for distributed Event Pattern matching.. The application should be able to run any number of processing nodes. For the needs of this thesis, example of 4 nodes will be sufficient.
}

%% The following section sets up the bibliography.
\usepackage{csquotes}
\usepackage[              %% When typesetting the bibliography, the
  backend=biber,          %% `numeric` style will be used for the
  style=numeric,          %% entries and the `numeric-comp` style
  citestyle=numeric-comp, %% for the references to the entries. The
  sorting=none,           %% entries will be sorted in cite order.
  sortlocale=auto         %% For more unformation about the available
]{biblatex}               %% `style`s and `citestyles`, see:
%% <http://mirrors.ctan.org/macros/latex/contrib/biblatex/doc/biblatex.pdf>.
\addbibresource{example.bib} %% The bibliograpic database within
                          %% the file `example.bib` will be used.
\usepackage{makeidx}      %% The `makeidx` package contains
\makeindex                %% helper commands for index typesetting.
%% These additional packages are used within the document:
\usepackage{paralist}
\usepackage{amsmath}
\usepackage{amsthm}
\usepackage{amsfonts}
\usepackage{url}
\usepackage{menukeys}
\begin{document}
\chapter{Úvod}
Cílem této práce je vytvoření  ...

\chapter{Zpracování událostí}
Se zvyšujícím se počtem zařízení, která jsou schopna produkovat data, se zvyšuje potřeba tato data analyzovat. Běžně rozšířeným způsobem je zpracování dat dávkově. Tedy, data se uloží a ve vhodnou dobu, typicky v noci, se analyzují.

Pokud však uvažujeme reálný provoz na síti, který se dnes v centrálních uzlech pohybuje okolo $1 Tb/s$ 
%%[https://is.muni.cz/do/rect/habilitace/1433/44368572/44368651/HP_kor.-verejna.pdf]
, je dávkové zpracování nereálné. Potřebujeme data analyzovat za běhu (angl. real time).

Jednotkou zpracování dat je událost (angl. event). Událost je základním pojmem používaným v oblasti zpracování událostí. Je definována jako objekt, který reprezentuje záznam o aktivitě v daném systému. Událost může mít vlastnosti. Typickým příkladem takové vlastnosti je čas vzniku události. [\ref{bib_1}] Jednoduchým příkladem události může být paket. Je to datová schránka, která obsahuje informace, které můžeme analyzovat. Samostatný paket nemá téměř žádnou vypovídající hodnotu, kdežto proud paketů je základem Internetu.

Takový proud událostí skrývá množství dat, která je možné získat až při komplexní analýze, která zohledňuje více událostí v řadě. To nazýváme \textit{komplexní zpracování dat (angl. complex event processing (CEP))}

\section{CEP}

TODO updavit:
What is Complex Event Processing? It is hard to capture all the fundamentals
of CEP in a single definition. At the beginning of his book David Luckham
defines CEP as just a set of techniques and tools which help to understand
and control an event-driven information system.

Jedná se o detekci a kontrolu v reálném čase na obrovských proudech dat. Jako problém se ale
začíná jevit přílišná roztříštěnost jednotlivých implementací i odlišné vnímání událostí v
nich. Aktuálně probíhají snahy o sjednocení [10]. Příkladem může být vznik Event
processing technical society (EPTS), která se zabývá rozvíjením základů CEP.
\section{Událost v CEP}
\section{Distribuované CEP}


\chapter{Závěr}
Here you can insert the appendices of your thesis.

\begin{thebibliography}{99}
\bibitem
LLUCKHAM, David. \textit{The Power of Events: An Introduction to Complex
Event Processing in Distributed Enterprise Systems.} Pearson
Education, Inc., 2002, ISBN 9780201727890 \label{bib_1}
\end{thebibliography}

\appendix %% Start the appendices.

\end{document}
